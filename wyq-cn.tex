\documentclass[11pt,a4paper]{moderncv}

% moderncv themes
%\moderncvtheme[blue]{casual}                 % optional argument are 'blue' (default), 'orange', 'red', 'green', 'grey' and 'roman' (for roman fonts, instead of sans serif fonts)
\moderncvtheme[green]{classic}                % idem
\usepackage{xunicode, xltxtra}
\XeTeXlinebreaklocale "zh"
\widowpenalty=10000

%\setmainfont[Mapping=tex-text]{文泉驿正黑}

% character encoding
%\usepackage[utf8]{inputenc}                   % replace by the encoding you are using
\usepackage{CJKutf8}
  
% adjust the page margins
\usepackage[scale=0.8]{geometry}
\recomputelengths                             % required when changes are made to page layout lengths
\setmainfont[Mapping=tex-text]{Hiragino Sans GB}
\setsansfont[Mapping=tex-text]{Hiragino Sans GB}
\CJKtilde

% personal data
\firstname{王逸群}
\familyname{}
\title{}               % optional, remove the line if not wanted

\mobile{13390782526}                    % optional, remove the line if not wanted
\email{wyqqrince@gmail.com}                      % optional, remove the line if not wanted
%% \quote{\small{``Do what you fear, and the death of fear is certain.''\\-- Anthony Robbins}}

\nopagenumbers{}

\begin{document}

\maketitle

\section{教育}
\cventry{2014至今}{硕士}{南京大学计算机科学与技术系,导师:马晓星}{将于 2017 年毕业}{}{}
\cventry{2010--2014}{本科}{南京大学计算机科学与技术系}{}{}{}  

\section{项目经历}
\renewcommand{\baselinestretch}{1.2}

\cventry{2014~至今}
{利用轻量级形式化工具解决复杂软件系统中的问题}
{Java}
{研究生科研项目}{}
{第一阶段是理论方面研究,使用形式化建模工具Alloy去解决验证分布式控制器正确性的问题,并提出一种模块化方法加速验证过程;第二阶段是利用该工具去解决云平台中配置管理问题,即对于一个应用场景,在验证其有效性的前提下,给出一套可行的配置方案,具体做法是利用ecore模型构建应用场景,然后使用该工具导入模型进行分析,最后通过调用封装好的chef接口来实现整个部署过程。整个研究项目是我一个人完成的。}

\vspace*{0.2\baselineskip}
\cventry{2014--2015}
{基于SCA规范的支持动态更新的中间件系统}
{Java}
{研究生科研项目}{}
{该系统是一个支持构件级别动态更新的中间件系统,它在Apache~Tuscany基础上扩展了对构件间动态依赖、事务的管理,支持构件应用的部署、运行和动态更新,同时兼容SCA规范,并对外提供了一组动态更新管理接口。在研究生阶段,我对该系统进行收尾工作,在了解其设计、功能等方面后,对系统实验进行了设计和开发,实验结果可以正确反映出不同更新算法的优势和特点。}

\vspace*{0.2\baselineskip}
\cventry{2014}
{KR自动机分解算法实现及复杂度分析}
{Java}
{本科毕设}{}
{该项目完成的是一个计算机理论方面的关于自动机分解算法的实现。在了解算法的基本原理和内容后,对数据结构和系统进行简单的设计,并且在项目开发的同时,进行了单元测试,功能测试等多种方法以保证系统的正确性和有效性。最后通过理论分析得到该算法的时间复杂度和空间复杂度,并利用实验数据对分析结果进行了验证。}

\vspace*{0.2\baselineskip}

\section{奖项}
\cventry{2013}{国家奖学金}{}{}{}{}
\cventry{2012}{高教社杯全国大学生数学建模竞赛本科组全国二等奖}{}{}{}{}
\cventry{2012}{董氏东方企业奖学金}{}{}{}{}
\cventry{2011}{人民奖学金三等奖}{}{}{}{}

\section{技能}
\cventry{语言}{Java > C++ > C}{}{}{}{}
\cventry{英语}{CET-6}{较为流利的英语读写能力和口语}{}{}{}

% \cvline{Photography}{\small Digital photography is my newest hobby.}

\closesection{}                   % needed to renewcommands
\renewcommand{\listitemsymbol}{-} % change the symbol for lists

\end{document}
