\documentclass[11pt,a4paper]{moderncv}
% moderncv themes
% \moderncvtheme[blue]{casual}                 % optional argument are 'blue' (default), 'orange', 'red', 'green', 'grey' and 'roman' (for roman fonts, instead of sans serif fonts)
\moderncvtheme[green]{classic}                % idem
\usepackage{xunicode, xltxtra}
\XeTeXlinebreaklocale "zh"
\widowpenalty=100000


%\setmainfont[Mapping=tex-text]{文泉驿正黑}

% character encoding
%\usepackage[utf8]{inputenc}                   % replace by the encoding you are using
\usepackage{CJKutf8}

% adjust the page margins
\usepackage[scale=0.82]{geometry}
\recomputelengths                             % required when changes are made to page layout lengths
\setmainfont[Mapping=tex-text]{Hiragino Sans GB}
\setsansfont[Mapping=tex-text]{Hiragino Sans GB}
\CJKtilde

\usepackage{fontspec}
\defaultfontfeatures{
    Path = /usr/local/texlive/2015/texmf-dist/fonts/opentype/public/fontawesome/ }
\usepackage{fontawesome}

% personal data
\firstname{程硕}
\familyname{}
\title{}               % optional, remove the line if not wanted

\mobile{18252063039}                    % optional, remove the line if not wanted
\email{njucshuo@gmail.com}                      % optional, remove the line if not wanted
\homepage{cshuo.top}
\extrainfo{\faGithub\ github.com/cshuo}
% \quote{\small{``Do what you fear, and the death of fear is certain.''\\-- Anthony Robbins}}

\nopagenumbers{}

\begin{document}

\maketitle

\section{教育}
\cventry{2015--至今}{硕士}{南京大学计算机科学与技术系,导师:曹春}{将于 2018 年毕业}{}{}
\cventry{2011--2015}{本科}{南京大学计算机科学与技术系}{}{}{}

% \section{实习经历}
% \cventry{2014.6--2014.9}
% {微软}{Android}{Bing Search on Android}{}
% {在实习期间,从零开始学习安卓,并能够迅速参与到项目开发中,主要负责开发新版本中的一些核心功能,如集成语音搜索,图片编辑,浏览器插件等。在实习过程中,能够与团队中成员(包括其他程序员,测试人员,客户经理,设计等)进行高效的沟通(包含英文),能够在短时间内为团队做出高质量的贡献。}

\section{技能}
\cventry{语言}{Python > Java > JS > C > C++}{}{}{}{}
\cventry{系统平台/框架}{Linux, Openstack, Ansible, Django, AngularJS}{}{}{}{}
\cventry{英语}{CET-6}{较为流利的英语读写能力和口语}{}{}{}

\section{项目经历}
\renewcommand{\baselinestretch}{1.2}

\cventry{2016--至今}
{IaaS云平台的动态资源调度}
{Python}{AngularJS}{}
{面向IaaS云平台的动态资源调度系统,根据IaaS云平台中虚拟机计算资源使用信息,基于负载预测和启发式装箱算法BFD,设计实现了动态资源调度策略,在降低IaaS云平台能耗以及保障应用SLA方面效果较好,核心技术思想已整理发表HPCC 2016论文一篇及专利一个, 并在Openstack平台上以插件形式进行了实现。该项目由本人负责设计实现。}

\vspace*{0.2\baselineskip}
\cventry{2015}
{面向关联性资源的分布式监控系统}
{Python}
{}{}
{系统实现的是基于被监控资源语义信息的分布式监控技术, 通过构建资源本体模型来表现资源之间的关联关系,基于资源本体语义及本体推理分析,实现对分布式应用计算资源的联动监控,以及性能异常时对资源使用的故障定位。本人是项目主要负责人。}

\vspace*{0.2\baselineskip}
\cventry{2015--2016}
{软件定义的容器云平台}
{Python}
{Django}{}
{以“软件定义”为核心思想的容器云计算平台。基于容器技术、Overlay网络技术、分布式存储技术等,实现了特定应用所依赖的计算资源以及运行环境的定制,支持应用的快速部署,可靠运行和高效维护。本人负责项目的前端开发以及后端Restful接口的设计实现。}

\vspace*{0.2\baselineskip}


\section{奖项}
\cventry{硕士}{2016 英才奖学金二等奖}{}{}{}{}
\cventry{硕士}{2015 学业奖学金一等奖}{}{}{}{}
% \cventry{本硕}{2015 保送南大计算机系研究生}{}{}{}{}
\cventry{本科}{2012-2014 连续获得人民奖学金二等奖}{}{}{}{}


\section{发表论文}
\cventry{2016}
{\textbf{Shuo Cheng}\textnormal{, Chun Cao, Ping Yu, and Xiaoxing Ma}}
{ SLA-Aware and Green Resource Management of IaaS Clouds}
{ In HPCC'2016}{}{}


% \section{其他}
% \cventry{博客}{http://cshuo.top}{}{}{}{}
% \cventry{Github}{https://github.com/cshuo}{}{}{}{}
% \cvline{Photography}{\small Digital photography is my newest hobby.}

\closesection{}                   % needed to renewcommands
\renewcommand{\listitemsymbol}{-} % change the symbol for lists

\end{document}
